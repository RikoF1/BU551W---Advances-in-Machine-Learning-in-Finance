\documentclass[11pt,a4paper]{report}
\fontfamily{cmss}
\hfuzz=9999pt % "fix" hbox overfull
\usepackage{hyperref}
\hypersetup{
    colorlinks=true,
    linkcolor=blue,
    filecolor=magenta,      
    urlcolor=blue
}

\begin{titlepage}
\title{BU551W - Advances in Machine Learning in Finance \\ Lecture Notes}
\author{Rodrigo Miguel}
\date{\today}
\end{titlepage}

\begin{document}
\maketitle
\tableofcontents

\chapter{Lecture 1 - Introduction to Machine Learning in Finance}
\section{Big Data in Finance}
\subsection{What is Big Data?}
\paragraph{Definition} Volume (the amount of data), velocity (the speed of data processing) and variety (the number of types of data).
\subsection{Big Data Applications}
\begin{itemize}
    \item Analyse customer satisfaction;
    \item Speed up manual processes with transactional data;
    \item Analyse financial performance;
    \item Customer recommendation.
\end{itemize}
\subsection{Big Data in Finance}
From a \textbf{practitioners'} perspective, big data refers to petabytes of structure and unstructured data that can be used to anticipate customer behaviour and create strategies for banks and other financial institutions.
\\
From an \textbf{academic} perspective, big data in finance research is large sized, high dimension and with a complex structure:
\begin{itemize}
    \item Large size: the data is large in an absolute and relative sense, e.g. transaction-level market microstructure data;
    \item High dimension: the data has many variables relative to its sample size;
    \item Complex structure: unstructured data includes text, pictures, videos, audio and voice.
\end{itemize}
\subsection{Big Data Challenges in Finance}
\begin{itemize}
    \item \textbf{Regulatory framework:} strict regulatory requirements that need to be followed, e.g. Fundamental Review of the Trading Book, governing access to critical data and demands accelerated reporting.
    \item \textbf{Data privacy/security:} unauthorised data collection, hackers, etc;
    \item \textbf{Data quality:} ensuring information is accurate, usable and secure is a challenge;
    \item Cloud solutions for financial data.
\end{itemize}

\section{The Role of Machine Learning}
\subsection{What is Machine Learning?}
\paragraph{Machine Learning} Is a field of study that allows computers the ability to learn and improve without being programmed continuously.
\\Traditional programming is done by a programmer that programs a computer and runs the software to get an output. To improve it, more programming needs to be done by the programmer.
Machine learning reads inputs and outputs, the computer automatically processes these and outputs a program.
\subsection{Applications of Machine Learning}
\begin{itemize}
    \item E-mail spam detection;
    \item Face detection and matching, e.g. Face ID;
    \item Self-driving cars;
    \item Language translation;
    \item Drug design;
    \item etc.
\end{itemize}
\subsection{Machine Learning in the field of Finance}
In 2020, a definition of machine learning was proposed:
\begin{itemize}
    \item A diverse collection of high-dimensional models for statistical prediction;
    \item "Regularisation" methods for model selection and mitigation of overfit;
    \item Efficient algorithms for searching among a vast number of potential specifications.
\end{itemize}
\subsection{How Machine Learning is Used in Finance}
\begin{itemize}
    \item Fraud detection and prevention: machines learn to identify and combat fraudulent financial transactions;
    \item Robo-advisors;
    \item Credit rating;
    \item Algorithmic trading;
    \item Return prediction.
\end{itemize}
\subsection{Terminology}
\begin{itemize}
    \item \textbf{Algorithm:} set of rules that a machine follows to achieve a certain goal. Example: Cooking recipes are algorithms, the cooking steps are algorithms instructions, and ingredients are inputs. The cooked food is the output.
    \item A \textbf{Learner} or \textbf{Machine Learning Algorithm} is a program used to learn a machine learning model from data.
    \item A \textbf{Black Box Model} is a system that does not reveal its internal mechanisms, opposite of a \textbf{White Box};
    \item \textbf{Interpretable Machine Learning} refers to the methods and models that make the behaviour and predictions of the machine learning systems understandable to humans;
    \item A \textbf{Dataset} is a table with the data from which the machine learns from;
    \item An \textbf{Instance/Observation} is a row in the dataset;
    \item The \textbf{Features} are the inputs used for prediction or classification (column in the dataset).
    \item A \textbf{Machine Learning Task} is the combination of a dataset with features and a target. E.g. classification, regression, survival analysis and/or clustering;
    \item The \textbf{Prediction} is what the machine learning model "guesses" what the target value should be, based on the features given.
\end{itemize}

\chapter{Lecture 2 - Financial Data Science with Python}
\section{Python Infrastructure}
Python is well suited at addressing technological challenges in the financial industry as well as in financial data analytics.
\section{Basic Data Types}
\begin{itemize}
    \item Integers;
    \item Floats;
    \item Booleans;
    \item Strings;
    \item Printing and String Replacements;
    \item import re;
    \item Tuple;
    \item List;
    \item Dictionary;
    \item Set;
\end{itemize}
\section{Numerical Computing with NumPy}
\begin{itemize}
    \item Arrays of Data;
    \item Regular NumPy Arrays;
    \item Structured NumPy Arrays;
\end{itemize}
\section{Data Analysis with pandas}
\begin{itemize}
    \item DataFrame Class;
    \item Basic Analytics;
    \item Visualisation;
\end{itemize}
\section{Python with Financial Data}
\begin{itemize}
    \item Retrieve Stock Price Data (EOD) for certain instruments;
    \item Work with Data using \textit{numpy} and \textit{pandas}
    \item Look up information about the data stored;
    \item Select columns and rows from a "DataFrame" object;
    \item Visualize data stored in a "DataFrame object;
    \item Vectorial calculations;
    \item Generate dummy data that resembles financial time series data;
    \item Store data in binary format on disk;
    \item Export data to a CSV file and read it.
\end{itemize}
\section{Trading Strategy Basics}
\begin{itemize}
    \item Generate trading signals in vectorized fashion;
    \item Calculate and visualize the cumulative returns of any trading strategies.
\end{itemize}
\section{Portfolio}
\begin{itemize}
    \item Calculate the return of a portfolio consisting of multiple assets;
    \item Calculate the portfolio variance by using the covariance matrix.
\end{itemize}




\chapter{Lecture 3 - Web Scraping with Python}
\chapter{Lecture 4 - Machine Learning Basics}
\chapter{Lecture 5 - Supervised Learning}
\chapter{Lecture 6 - Supervised Learning - Classifiers}
\chapter{Lecture 7 - Unsupervised Learning and Cross-Validation}
\chapter{Lecture 8 - Text as Data}
\chapter{Lecture 9 - Financial Technology}
\chapter{Lecture 10 - Fully Automated Trading System}

\end{document}